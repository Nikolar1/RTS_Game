\documentclass[11pt,a4paper]{article}

\usepackage[serbianc]{babel}
\title{Семинарски рад Ц развој компјутерских игара}
\author{Никола Радојчић 62/19}
\usepackage[utf8]{inputenc}

\begin{document}

%naslov
\maketitle
\newpage

\tableofcontents
\newpage

%telo dokumenta
\section{Кратки опис}
Стратешка игрица која се састоји искључиво од своје кампање која има структуру која је комбинација између кампања из игрица \emph{Faster than Light} и \emph{Homeworld} где су саме мисије сличне мисијама у игрици \emph{Stroghold}. Између мисија у кампањи се очувавају сви војници и ресурси који су у тој мисији сакупљени као у \emph{Homeworld}, док се мисије бирају на мапи на којој играч има избор путанје коју жели да узме да би стигао до циља. Саме мисије се састоје од постављанја кампа експлатације ресурса изградње помоћних зграда као и тренирања војника зарад победе над непријатељским снагама које се већ налазе на мапи пре доласка или се појављују после одређеног времена. Непријатељ се не развија током мисије. Игрица би требала да се дешава у средњем веку као у \emph{Stroghold}.

\section{Прича}
Идеја приче се врти око две зараћеме државе Држава1 (Играчева држава) и Држава2. Држава1 је скоро покорила нови регион који и даље није сасвим веран Држава1. Држава2 је ово ширење Држава1 до њених границасхватила као претњу и одлучила да нападне Држава1 под изговором ослобађања покорених народа над којима влада Држава1. Кампања почиње после катастрофалне битке где је већина војске Држава1 побијено и играч мора да води остатке војске који се повукао назад до оригиналне границе Држава1 да би се спојио са својим појачањима у утврђеном граду Град1. Увидевши насталу славост Држава1 покорени регион се диже на буну и придружује Држава2 у нади да ће остварити слободу и сада остаци армије Држава1 морају да направе себи пут кроз побуњени регион док им је Држава2 за петама. Циљ кампање је да се победи армија Држава2 без обзира да ли се играч повезао са појачањима

\section{Кампања}
Кампања ће се одржавати на мапи на којој постоји више путева којима се може достићи утврђени град сваки пут води до неког чвора на ком ће се одиграти мисија. Чвор може представљати једну од три типа мисија опсада, битка на пољу или интерактивна мисија.

При кретању на мапи играчева армија троши своје залихе хране које је скупила у мисијама па играч не може да одабере само најлакши пут преко интерактивних мисија већ мора понекад да одигра и друга два типа мисија. Такође играч који изабере најлакши пут до града се може наћи у неизлазној ситуацији при доласку у Град1 јер појачања која тамо чекају нису довољна да се победи непријатељска армија.

Непријатељска армија се креће спорије него играчева армија по мапи због своје величине али она увек прати најкраћи пут до Град1 и ако престигне играчеву армију она ће почети опсаду града која ће бити симулирана у позадини па неће бити сигурно када ће непријатељ освојити град, играч има шансу да се укључи у опсаду у ком случају ће бранитељи града изаћи из зидина и напасти непријатеља заједно са играчем. Ако Град1 падне играч ће бити онај који мора да га опседа али постојаће одређен временски период после пада града у којем ће зидине бити оштећене олакшавајући опсаду за играча. Када непријатељска армија дође до града који није био опседан она расте у велични захваљујући појачањима из града. Како расту обе армије успоравају.
\subsection{Опсада}
Опсада може да буде опсада над селом у опсада над градом. при опсади над селом играч треба да освоји центар села које може а и немора бити ограђено дрвеним зидом који се може срушити, док ће при опсади града непријатељ имати више војника и камене зидове који се никако немогу срушити те ће војници морати да се пењу на њих, али ће и  награда бити већа. Награда за успешну опсаду се састоји од свих ресурса који се могу покупити око насеља  који ће наравно бити много обилнији него у другим мисијама и од крајње награде у виду велике количине ресурса и људи који су били заробљени у граду па се придружују војсци и повећавају популацију.
\subsection{Битка на пољу}
Овај тип мисије се одиграва тако што је део непријатељске војске већ на мапи обично чувајући неке ресурсе. Играч треба да победи непријатеље који се налазе на тој мапи. Постоји могућност доласка појачања и оближњих градова да помогну непријатељу.
\subsection{Интерактивна мисија}
Овај тип мисије се одиграва у целости на кампањској мапи где играч добија избор одлука које ће утицати на кампању на различите начине, као на пример може да жртвује брзину кретања до следећег чвора али да добије неке ресурсе за узврат.

\section{Мисија}
Све мисије, осим интерактивних, се одигравају тако што се сви играчеви војници појављују на једном делу мапе заједно са вагоном који играч може да пострави на неку локацију да би поставио свој камп ако зграда која представља буде уништена играч је игубио игру. Сви војници и радници се праве у кампу и захтевају популацију и ресурсе. Ресурсе скупљају радници и носе их до најближе локације за остављање ресурса  био то сам камп или вагон за ресурсе који се у њему прави. Популација зависи од до сада сакупљене популације  у кампањи и неможе се повећати током мисије осим у неким специјалним случајевима или распуштањем неког војника или радника. када се направи војник или радник он троши популацију која се неће вратити ако он умре. Када мисија потраје предуго непријатељска армија може сустићи играча и полако почети да улази у битку са истог места одакле је играч дошао и у том тренутку играч има кратко време да побегне из мисије завршавајући циљ мисије и повлачећи се после. При битци на пољу војници главне непријатељске армије се не рачунају у захтеве мисије. Време потребно да се непријатељска армија укључи у мисију се рачуна на основу њене удаљености на кампањској мапи.

\section{Приказ игре}
Током гледања кампањске мапе она треба да изгледа као старе мапе тог периода, то јест претежно се концентришући на путеве, насеља и битне локације, који су представљени чворовима, док остатак мапе приказује свет али нацртан отприлике у односу на чворове.

Током мисија играч види уближену мапу која представља терен на том чвору. Терен је представљен светилијим и тамнијим бојама у зависности на своју висину а остатак ствари на мапи је приказан одговарајућим сличицама. 

Идеја је да играч има утисак да се налази у командном шатору и на мапи гледа позиције своје и непријатељске војске и оданде издаје наређења самим тим све мапе као и игрица у целини приказани су у 2Д.



\end{document}