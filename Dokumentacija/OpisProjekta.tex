\documentclass[11pt,a4paper]{report}

\usepackage[serbianc]{babel}
\usepackage{placeins}

\title{Семинарски рад Ц развој компјутерских игара}
\author{Никола Радојчић 62/19}
\usepackage[utf8]{inputenc}
\usepackage[hidelinks]{hyperref}

\begin{document}

%naslov
\maketitle
\newpage

\tableofcontents
\newpage

%telo dokumenta
\chapter{Опис идеје игрице}

\section{Кратки опис}
Стратешка игрица која се састоји искључиво од своје кампање која има структуру која је комбинација између кампања из игрица \emph{Faster than Light} и \emph{Homeworld} где су саме мисије сличне мисијама у игрици \emph{Stroghold}. Између мисија у кампањи се очувавају сви војници и ресурси који су у тој мисији сакупљени као у \emph{Homeworld}, док се мисије бирају на мапи на којој играч има избор путанје коју жели да узме да би стигао до циља. Саме мисије се састоје од постављанја кампа експлатације ресурса изградње помоћних зграда као и тренирања војника зарад победе над непријатељским снагама које се већ налазе на мапи пре доласка или се појављују после одређеног времена. Непријатељ се не развија током мисије. Игрица би требала да се дешава у средњем веку као у \emph{Stroghold}.

\section{Прича}
Идеја приче се врти око две зараћене државе Држава1 (Играчева држава) и Држава2. Држава1 је скоро покорила нови регион који и даље није сасвим веран Држава1. Држава2 је ово ширење Држава1 до њених границасхватила као претњу и одлучила да нападне Држава1 под изговором ослобађања покорених народа над којима влада Држава1. Кампања почиње после катастрофалне битке где је већина војске Држава1 побијено и играч мора да води остатке војске који се повукао назад до оригиналне границе Држава1 да би се спојио са својим појачањима у утврђеном граду Град1. Увидевши насталу славост Држава1 покорени регион се диже на буну и придружује Држава2 у нади да ће остварити слободу и сада остаци армије Држава1 морају да направе себи пут кроз побуњени регион док им је Држава2 за петама. Циљ кампање је да се победи армија Држава2 без обзира да ли се играч повезао са појачањима

\section{Кампања}
Кампања ће се одржавати на мапи на којој постоји више путева којима се може достићи утврђени град сваки пут води до неког чвора на ком ће се одиграти мисија. Чвор може представљати једну од три типа мисија опсада, битка на пољу или интерактивна мисија.

При кретању на мапи играчева армија троши своје залихе хране које је скупила у мисијама па играч не може да одабере само најлакши пут преко интерактивних мисија већ мора понекад да одигра и друга два типа мисија. Такође играч који изабере најлакши пут до града се може наћи у неизлазној ситуацији при доласку у Град1 јер појачања која тамо чекају нису довољна да се победи непријатељска армија.

Непријатељска армија се креће спорије него играчева армија по мапи због своје величине али она увек прати најкраћи пут до Град1 и ако престигне играчеву армију она ће почети опсаду града која ће бити симулирана у позадини па неће бити сигурно када ће непријатељ освојити град, играч има шансу да се укључи у опсаду у ком случају ће бранитељи града изаћи из зидина и напасти непријатеља заједно са играчем. Ако Град1 падне играч ће бити онај који мора да га опседа али постојаће одређен временски период после пада града у којем ће зидине бити оштећене олакшавајући опсаду за играча. Када непријатељска армија дође до града који није био опседан она расте у велични захваљујући појачањима из града. Како расту обе армије успоравају.
\subsection{Опсада}
Опсада може да буде опсада над селом у опсада над градом. при опсади над селом играч треба да освоји центар села које може а и немора бити ограђено дрвеним зидом који се може срушити, док ће при опсади града непријатељ имати више војника и камене зидове који се никако немогу срушити те ће војници морати да се пењу на њих, али ће и  награда бити већа. Награда за успешну опсаду се састоји од свих ресурса који се могу покупити око насеља  који ће наравно бити много обилнији него у другим мисијама и од крајње награде у виду велике количине ресурса и људи који су били заробљени у граду па се придружују војсци и повећавају популацију.
\subsection{Битка на пољу}
Овај тип мисије се одиграва тако што је део непријатељске војске већ на мапи обично чувајући неке ресурсе. Играч треба да победи непријатеље који се налазе на тој мапи. Постоји могућност доласка појачања и оближњих градова да помогну непријатељу.
\subsection{Интерактивна мисија}
Овај тип мисије се одиграва у целости на кампањској мапи где играч добија избор одлука које ће утицати на кампању на различите начине, као на пример може да жртвује брзину кретања до следећег чвора али да добије неке ресурсе за узврат.

\section{Мисија}
Све мисије, осим интерактивних, се одигравају тако што се сви играчеви војници појављују на једном делу мапе заједно са вагоном који играч може да пострави на неку локацију да би поставио свој камп ако зграда која представља буде уништена играч је игубио игру. Сви војници и радници се праве у кампу и захтевају популацију и ресурсе. Ресурсе скупљају радници и носе их до најближе локације за остављање ресурса  био то сам камп или вагон за ресурсе који се у њему прави. Популација зависи од до сада сакупљене популације  у кампањи и неможе се повећати током мисије осим у неким специјалним случајевима или распуштањем неког војника или радника. када се направи војник или радник он троши популацију која се неће вратити ако он умре. Када мисија потраје предуго непријатељска армија може сустићи играча и полако почети да улази у битку са истог места одакле је играч дошао и у том тренутку играч има кратко време да побегне из мисије завршавајући циљ мисије и повлачећи се после. При битци на пољу војници главне непријатељске армије се не рачунају у захтеве мисије. Време потребно да се непријатељска армија укључи у мисију се рачуна на основу њене удаљености на кампањској мапи.

\section{Приказ игре}
Током гледања кампањске мапе она треба да изгледа као старе мапе тог периода, то јест претежно се концентришући на путеве, насеља и битне локације, који су представљени чворовима, док остатак мапе приказује свет али нацртан отприлике у односу на чворове.

Током мисија играч види уближену мапу која представља терен на том чвору. Терен је представљен светилијим и тамнијим бојама у зависности на своју висину а остатак ствари на мапи је приказан одговарајућим сличицама. 

Идеја је да играч има утисак да се налази у командном шатору и на мапи гледа позиције своје и непријатељске војске и оданде издаје наређења самим тим све мапе као и игрица у целини приказани су у 2Д.

Услед не могућности цртања задовољвајућег чекића за икону радника донета је одлука да уметност буде \emph{Pixel art}.

\section{Јединице}
Све јединице у игрици имају исте података који их карактеришу а разликују се само по вредностима. Јединице можемо поделити у две категорије на остову типа популације који је потребан за њихово регрутовање. Постоје јединице прве и друге класе, јединице прве класе долазе од популације ратника а јединице друге класе долазе из дела популације способне за борбу. Популација осим ова два дела има и део неспособан за рат који само успорава армију у кампањи и који је сачињен од људи које играч може да покупи током кампање, њихово купљење ће донети бонусе на почетку али ће они остати са играчем до краја игре. Свака јединица може припадати или играчу или непријатељу, и она има своју цену која представља потребну опрему за опремање те јединице. Свака јединица може да се туче али њена ефективност у тучи зависи од њене јачине и способности пробоја оклопа у тучи. Неке од јединица имају способност борбе на даљину у том случају ако нису у контакту са непријатељем и ако се непријатељ налази у њиховом домету они за напад не користе своје вредности за борбу у тучи, већ се ослањају на своју јачину и пробој оклопа на даљину, а прелазе на вредности за тучу при контакту са непријатељем. Јединице које гађају имају и своју брзину гађања као и прецизност која се користи при прорачунима да би се одредило да ли јединица чини икакву штету. Свака јединица има вредности за оклоп одбрану и своје здравље. Оклоп директно смањује штету коју наноси непријатељска јединица, то јест он се одузима од непријатељског напада и ако та вредност падне на нулу јединица неће претрпети никакву штету. Пре него што се оклоп одузме од непријатељског напада, непријатељски пробој оклопа умањује оклоп по следећој формули:
\\ о*(1-по/100)
\\ о-оклоп 
\\ по-пробој оклопа
\\ Одбрана умањује непријатељски напад на исти начин ко што пробој оклопа умањује оклоп али се она умањује непријатељски напад тек када се од њега одузме оклоп. Здравље јединице одређује колико штете јединица може да претрпи, када здравље дође до нуле јединица умире. Штета се рачуна тако што се од здравља одузме непријатељски напад после свих пређашњих прорачуна. Постоји седам врста јединица, а то су: радници, копљани, мачеваоци, стрелци, праћкаши, аркуебусијери и коњаници.

\subsection{Јединице друге класе}
Јединице друге класе чине радници, копљаници, праћкаши и аркуебусијери. Све јединице друге класе могу да сакупљају ресурсе и да граде палисаде. Радници су најпростији тип јединица они немогу да гађају а уједно су и најгори у тучи, али зато су најјефтинији и намењени су за скупљање ресурса. Копљани су намењени за тучу али су најслабији у том погледу ипак они су и даље јефтини и представљају једину јединицу намењену за тучу у јединицама друге класе. Копљани уједно имају и бонус против коњаника који смањује разлику између њих али је коњаник и даље јачи. Ако се утроши иста количина ресурса на прављење копљаника и коњаника, копљаници ће победити. Праћкаш је најјефтинија јединица која може да гађа њихов напад је слаб и немају никакав пробој оклопа али имају највећи домет. Аркубусијери користе примитивне пушке које јако споро пуцају и имају очајну прецизност па самим тим и мали домет, тако да ова јединица има најгори домет, прецизност и брзину гађања, такође су веома скупи због цене пушака, али због лакоће тренирања они спадају у другу класу и пошто користе пушке имају највећи пробој оклопа од свих јединица, њихов пробој оклопа је толико велик да они игноришу непријатељски оклоп. Аркубусијери не могу да пуцају у непријатеља ако се нека јединица налази између њих и тог непријатеља, док све остале јединице са способношћу гађања могу да гађају преко других јединица и брину само о свом домету.

\subsection{Јединице прве класе}
Јединице прве класе чине  мачеваоци, стрелци и коњаници. Јединице које припадају другој класи долазе од ратника и не може им се наредити да сакупљају ресурсе, самим тим они немају никакву сврху ван борбе. Мачеваоци су веома добри у тучи имају јак оклоп и добру одбрану, они поред коњаника поседују највећи напад и пробој оклопа захваљујући резервном оружју али имају и најмању брзину у односу на све јединице и уједно су друга најскупља јединица. Стрелци имају способност гађања и имају највећу брзину и прецизност гађања док су њихови напад и пробој оклопа на даљину мањи од оног који аркубусијери поседују они ће засигурно у одређеном временском периоду нанети више штете спрам аркубусијера. Стрелци су најбољи у тучи од свих јединица са способношћу гађања захваљујући њиховој вештини. Коњаници су најбоља јединица у тучи имајући све предности мачеваоца а још и највећу брзину као и највеће здравље у односу на све јединице, али су зато и најскупља јединица у игрици.

\subsection{Плаћеници}
Плаћеници се могу ангажовати током интерактивних мисија. Они су сачињени од јединица прве класе у предодређеним пропорцијама и не троше популацију, али ни не повећавају је када се распусте. Они су знатно скупљи од јединица које би биле нормално ангажоване. У случају да непријатељска армија сустигне играча, ако је та армија много већа од играчеве сви плаћеници ће издати играча и побећи јер му нису верни и не желе да погину.

\subsection{Опсадна опрема}
Сва опсадна опрема се гради искључиво од дрва. Сва опсадна опрема захтева да се њој прикључи јединица да би она радила, што значи да без јединице она је бескорисна. У опсадну опрему спадају мердевине, опсадна кула, јарац за пробој капије, катапулт и  опсадни самострел. Мердевине су најјефтинија врста опсадне опреме и оне дозвољавају јединици која их узме да се пење на зидове, али ће та јединица добити огромно смањење својих способности и своје брзине доке се буде пењала на зид. Опсадна кула као и мердевине дозвољава пењање на зидове али је смањење способности јединице веома мало код ње. Јарац служи искључиво за пробој капије и јединице га могу користити да пробију исту. Катапулти служе за пробој палисада, зидови се немогу пробити. Опсадни самострели се могу користити за пробој палисада и за гађање војника који се налазе на зидовима. Сва опсадна опрема има своје здравље и може бити уништена. Коњаници су једина јединица која неможе да узима опсадну опрему, самим ти они су једина јединица која неможе да се пење на зидове.

\begin{table}[h]
\subsection{Подаци о јединицама}
\centering
\begin{tabular}{|| c || c | c | c | c | c | c | c ||}
\hline
Тип & Рад & Коп & Мач & Стр & Пра & Арк & Коњ \\
\hline\hline
Ц & 50 & 100 & 200 & 100 & 75 & 200 & 400\\
\hline
НТ & 2,5 & 10 & 20 & 4 & 2,5 & 2,5 & 25\\
\hline
ПОТ & 0 & 0 & 50 & 0 & 0 & 0 & 50\\
\hline
НД & 0 & 0 & 0 & 8 & 2,5 & 10 & 0\\
\hline
ПОД & 0 & 0 & 0 & 70 & 0 & 100 & 0\\
\hline
ПРЕ & 0 & 0 & 0 & 90 & 70 & 40 & 0\\
\hline
ДОМ & 0 & 0.5 & 0 & 8 & 10 & 5 & 0\\
\hline
БГ & 0 & 0 & 0 & 0.25 & 0.5 & 1 & 0\\
\hline
ОК & 0 & 0 & 7 & 0 & 0 & 0 & 7\\
\hline
ОД & 0 & 10 & 50 & 20 & 0 & 0 & 50\\
\hline
ЗДР & 25 & 25 & 50 & 50 & 25 & 25 & 75\\
\hline
БРЗ & 1 & 1 & 0,3 & 1 & 1 & 1 & 3 \\
\hline

\end{tabular}
\caption{Ц - Цена, НТ- Напад за тучу, ПОТ - Пробој оклопа у тучи, НД - Напад на даљину, ПОД - Пробој оклопа на даљину, ПРЕ - Прецизност, ДОМ - Домет, БГ - Брзина гађања, ОК - оклоп, ОД - Одбрана, ЗДР - Здравље, БРЗ - Брзина }
\end{table}
%sprecava da se tabela spusti nize
\FloatBarrier

\section{Ресурси}
Постоје три типа ресурса дрва, злато и залихе. Дрва се сакупљају из шума током мисија и она су једини тип ресурса који је не преноси на кампањску мапу она служе за изградњу зидова и кула као и за унапређивање кампа. Дрва су такође потребна и за изградњу опсадне опреме као што су катапулти, опсадни самострели, јарчеви за пробој капије и други. Сва сакупљена дрва се на крају мисије остављају, то јест губе. Злато је најважнији ресурс оно се користи за куповину војника, унајмљивање плаћеника а потребно је и за унапређивање кампа. Злато се корисити и у понекој интерактивној мисији. У већини мисија постоје рудници злата и на крају сваке мисије постоји награда у злату. Залихе престављају све ресурсе потребне за живот кампа и оне се временом троше, на кампањској мапи је то очигледно током дугих маршева али оне полако опадају и током битке. Када играч остане без залиха он губи игру ако се то десило на кампањској мапи, а ако се то десило током мисије играч добија одређену количину времена да заврши ту мисију у нади да ће награда у залихама бити довољна да му помогне. Залихе се добијају на крају сваке мисије, у случају да је играч остао без залиха током мисије награда се повећава без знања играча, да би било више случајева где се играч за длаку спасао пораза.

\chapter{Опис кода игрице}

\section{Одабир објеката}
Одабир јединица се обавља преко три класе \emph{PlayerManager}, \emph{MultiSelect} и \emph{InputHandler}. Класа \emph{PlayerManager} садржи промењиву која показује на над објекат свих играчевих јединица и позива током сваког фрејма  функцију \texttt{HandleUnitMovment()} из класе \emph{InputHandler}. \emph{MultiSelect} је помоћна класа која у себи садржи функције које се касније користе при одабиру више јединица превлачењем. Улаз од корисника се обрађује у класи \emph{InputHandler}. Функције \texttt{SelectUnit(Transform,bool)} и \texttt{DeselectUnits()} у класи \emph{InputHandler} одрађују одабир већ одређеног објекта и брисање пређашњих одабира респективно. Одабир већ одређеног објекта у функцији \texttt{SelectUnit{Transform,bool}} се одвија тако што се прво провери да ли је други параметар тачно или нетачно и у случају да је нетачно позива се функција за брисање пређашњих одабира јер други параметар говори да ли смеју да се одаберу више објеката. После провере другог параметра прослеђена трансформација која представља локацију јединице се додаје у глобалну промењиву која чува податке о одабраним објектима. На крају се у одабраном објекту проналази под објекат са именом \emph{Highlight} и она се активира то јест одкрива на екрану. Када се пређашњи одабири бришу по позиву функције \texttt{DeselectUnits()} пролази се кроз лису пређашњих одабира и сваки се деактивира то јест сакрива на екрану, и на крају се избрише све из те листе. 

\subsection{Одређивање објеката за одабир кликтањем}
При притиску левог дугмета на мишу у функцији \texttt{HandleUnitMovment()} испуњава се услов прве иф петље и у њој се онда преузима локација миша и потом претвара у  одређену тачку у свету игрице, потом се проверава да ли се неки објекат налази на тој тачци. Ако се на тачци налази објекат у осмом слоју који представља јединице позива се функција за одабир јединица и прослеђује му се као први параметар трансформација погођеног објекта, а као други параметар резултат провере да ли је једно од \emph{Control(Ctrl)} дугмади притиснуто. Ако ни један од жељених слојева није погођен бришу се сви стари одабири као и у случају да ништа није погођено. 

\subsection{Одређивање јединица за одабир превлачењем}
Када год се притисне лево дугме миша увек се једна глобална промењива постављан на тачно а када се то дугме пусти промењива се постављан на нетачно. До год је та промењива постављена на тачно исцртава се на екрану правоугаоник од почетне до тренутне тачке помоћу класе \emph{MultiSelect}. Када се дугме пусти пролази се кроз сву децу над објекта који се налази у \emph{PlayerManager} и онда кроз сву децу те деце што представља јединице па се проверава да ли се та јединица налази у распону правоугаоника, ако се налази позива се функција за одабир објекта са и прослеђује се та јединица и тачно. На крају се промењива враћа на нетачно.

\subsubsection{Цртање правоугаоника и одређивање његове границе}
текст 

\section{Кретање}
За проналажење пута користи се \emph{A* Pathfinding Project} овај додатак обрађује мапу и проналази пут од објекта до његовог циља избегавајући препреке.

\section{Борба}
Функције за прорачун борбе се налазе у статичкој класи \emph{Combat}.

\section{Зграде}

\begin{thebibliography}{9}
\bibitem{UnityLearn}
\href{https://learn.unity.com/}{Unity Learn}
\bibitem{LithexProductions}
   Lithex Productions, \href{https://www.youtube.com/watch?v=oVM_ugro6Mw&list=PLQPhaRCbpx5U0kcamApy727XC0v1bF0PK&index=1}{How to BUILD and LAUNCH an RTS Game in UNITY}.
\bibitem{A*PathPro}
Aron Granberg, \href{https://arongranberg.com/astar/front}{A* Pathfinding Project}
\end{thebibliography}
\end{document}